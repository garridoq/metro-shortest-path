\documentclass{article}
\usepackage[francais]{babel}
\usepackage[utf8]{inputenc}
\usepackage{xcolor}
\usepackage[pdftex]{graphicx}
\usepackage{listings}
\usepackage{amsmath}
\usepackage[a4paper,includeheadfoot,margin=2.54cm]{geometry}
\usepackage{amsfonts}
\usepackage{fancyhdr}
\usepackage{titling}
\usepackage{algorithm}
\usepackage{algpseudocode}
\usepackage{hyperref}

\pagestyle{fancy}
\fancyhf{}
\fancyhead[LE,RO]{\theauthor}
\fancyhead[RE,LO]{\thetitle}
\fancyfoot[CE,CO]{\leftmark}
\fancyfoot[LE,RO]{\thepage}

\usepackage[thinlines]{easytable}

\title{Atelier d'approfondissement en informatique: Graphes et Algorithmes}
\author{Quentin Garrido}
\date{21 avril 2019}

\begin{document}

\maketitle
\tableofcontents
\pagebreak

%=============================================================================%
\section{Introduction}
\subsection{Objectifs}

L'objectif de ce projet est d'implémenter des algorithmes de recherche de plus courts
chemins dans un graphe, et plus particulièrement dans un graphe représentant le réseau
de métro de Paris.\\
Nous implémenterons tout d'abord l'algorithme de Dijkstra, puis le modifierons pour
qu'ils deviennent l'algorithme A* et nous optimiserons son temps d'éxécution grâce
à des structures de données adaptées.

\subsection{Utilisation}

Tout le code source est disponible à l'adresse suivante: \textit{https://github.com/garridoq/metro-shortest-path}.\\

Tout les éxécutables devraient vous être fournis dans le mail et devraient fonctionner sans devoir
les recompiler. Dans le cas contraire voici la démarche à suivre:\\

Un makefile est fourni pour la compilation, il servira à compiler les bibliothèques et les tests.
Une fois le code source obtenu il faudra exécuter la commande suivante pour compiler les bibliothèques:
\begin{lstlisting}[language=bash]
	> make
\end{lstlisting}

Vous pourrez alors compiler tous les fichiers de tests de la manière suivante:
\begin{lstlisting}[language=bash]
	> make NOM.exe
\end{lstlisting}
Où NOM est le nom du fichier de test (pour test\_heap.c, il faudra entrer make test\_heap.exe).\\

Voici la liste des fichiers de test et leur utilisation:
\begin{itemize}
	\item test\_heap:./test\_heap.exe , ce fichier permet de tester l'implémentation du tas
		  binaire et de ses opérations primaires.
	\item test\_dijkstra:./test\_dijkstra.exe GRAPH DEBUT FIN , ce fichier va récupérer
		  le graphe dans le fichier GRAPH, puis calculer le plus court chemin de DEBUT à FIN
		  en utilisant l'agorithme de Dijkstra, l'A* et A* avec une file de priorité.\\
		  Un fichier EPS sera créé pour chaque algorithme.\\
\end{itemize}

À titre de référence, voici à quoi ressemble le graphe entier du réseau de métro, que nous
utiliserons par la suite:
Les stations sont réprésentées par les sommets et nous avons une chaîne entre deux sommets
si ils sont reliés par le métro.

\begin{figure}[!hbt]
	\centering
	\includegraphics[width=\textwidth]{metro.eps}
	\caption{Graphe de référence du métro}
	\label{metro}
\end{figure}

\clearpage
%=============================================================================%
\section{Algorithme de Dijkstra}
\subsection{Implémentation}

Pour calculer le chemin de plus court d'un point D à A nous allons utiliser 
la version suivante de l'algorithme de Dijkstra, adaptée depuis le cours de 
l'unité Graphes et Algorithmes.\\

Ici nous n'avons pas besoin de calculer les chemins de notre sommet de départ
vers tous les autres sommets du graphe et nous nous arrêterons donc dès
que nous atteignons notre sommet d'arrivée.

\begin{algorithm}
\caption{Algorithme de Dijkstra}\label{dijkstra}
\begin{algorithmic}[1]
\Procedure{DIJKSTRA}{$E, \Gamma, l, d \in E, a \in E$}
	\State S = \{d\}, $\pi(d)$ = 0, k = 1, $x_1$ = d
	\ForAll{$x \in E$\textbackslash$ \{d\}$}
		\State $\pi(x) = \infty$
	\EndFor
	
	\While{$k < n$ et $\pi(x_k) < \infty$ }
		\ForAll{$y \in \Gamma(x_k) $ tel que $y \not\in S$}
			\State $\pi(y) = $ min[$\pi(y), \pi(x_k) + l(x_k, y)$]
		\EndFor
		\State Extraire $x \not\in S$ tel que $\pi(x)$=min$\{\pi(y), y \not\in S\}$
		\State k = k + 1, $x_k = x$, S = S $\bigcup \{x_k\}$
		\If{$x_k = a$}
			\State \textbf{break}
		\EndIf
	\EndWhile
	
	\State \textbf{return} $\pi$, S
\EndProcedure
\end{algorithmic}
\end{algorithm}

Nous implémentons S avec un tableau de $n = \vert E\vert$ éléments, correspondants aux sommets 
de notre graphe.\\
Ainsi nous l'initialiserons entièrement à 0 et S = S $\bigcup \{x_k\}$
correspondra à faire S[k] = 1.\\
Bien que cette implémentation soit plus coûteuse en mémoire qu'une liste chaînée
elle permettra d'implémenter l'appartenance à S en temps constant, opération très
utilisée aux lignes 7 et 10.\\
De plus l'ajout d'un élément sera aussi simplifié car nous n'aurons pas à vérifier
l'appartenance avant de l'insérer ou non.\\

\subsection{Résultats}

Considérons un trajets des stations Grande arche de la Défense (130) à Place d'Italie (244).
Nous trouvons alors le chemin le plus court suivant:



\pagebreak
%=============================================================================%
\section{Stratégie A*}

\subsection{Choix de l'heuristique}

\subsection{Implémentation de l'algorithme}

\begin{algorithm}
\caption{Algorithme A*}\label{astar}
\begin{algorithmic}[1]
\Procedure{heuristique}{$E,i \in E, a \in E$} \Comment{a est notre point d'arrivée}
	\State \textbf{return} distance\_euclidienne(i, a)
\EndProcedure
\\
\Procedure{A*}{$E, \Gamma, l, i \in E$}
	\State S = \{i\}, $\pi(i)$ = 0, k = 1, $x_1$ = i
	\ForAll{$x \in E$\textbackslash$ \{i\}$}
		\State $\pi(x) = \infty$
	\EndFor
	
	\While{$k < n$ et $\pi(x_k) < \infty$ }
		\ForAll{$y \in \Gamma(x_k) $ tel que $y \not\in S$}
			\State $\pi(y) = $ min[$\pi(y), \pi(x_k) + l(x_k, y)$]
		\EndFor
		\State Extraire $x \not\in S$ tel que $\pi(x)$=min$\{\pi(y)+$heuristique(y)$, y \not\in S\}$
		\State k = k + 1, $x_k = x$, S = S $\bigcup \{x_k\}$
	\EndWhile
	
	\State \textbf{return} $\pi$, S
\EndProcedure
\end{algorithmic}
\end{algorithm}

\pagebreak
%=============================================================================%
\section{Tas Binaire}

\subsection{Implémentation classique}

\subsection{Implémentation pour notre problème}

\subsection{Autre structures de données possibles}

\pagebreak
%=============================================================================%
\section{Affichage des chemins}

\end{document}
